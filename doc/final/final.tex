\documentclass[a4paper]{article}
%\documentclass[a4paper,titlepage]{article}
\usepackage{mystyle}

\begin{document}
    \maketitle

    \begin{abstract}
        During a 10 day period, a deployed estimator received new data and
        outcomes.

        The received data had significant differences from the data used in
        training, but the models was still able to produce a reasonable
        estimation of the outcomes.
        The AUC ROC score for the new outcomes was \(0.8752\).

        Using the new data and outcomes to retrain the data does not result
        in  very impressive improvement, but should progressively increase
        as more outcomes arrive.
    \end{abstract}


    \section{Intro}
    \label{sec:intro}
    In a previous report we analysed a dataset with characteristics of an
    unknown population and if they became unemployed in the next 12 months.
    We will assume knowledge of this report, if you are not familiar please
    read it.

    Several predictived models were tested and one was selected.
    This model was deployed online to receive further data during a 10 day
    period.
    In this time we received both more population samples and outcomes.

    In this report we analyse the behaviour of our deployed model in light
    of the new information.


    \section{Exploratory data analysis}
    \label{sec:data}

    During this 10 day period we received 9943 new samples and 498 new
    outcomes.
    In this section we will have a look at the new data and check for any
    anomalies relative to the original data used to train the model.

    Having a look at the data (\vref{sec:tables}) it's easy too spot that
    there are significant differences from the data used to train our model.
    We now have samples with \emph{eaned dividends} and different genders.
    The proportions of the categories for \emph{domestic relationship type}
    and \emph{domestic status} do not match.
    There are also some changes in \emph{job types} and \emph{profession}
    This is a considerable amount of change for our model to deal with.


    \section{Model Analysis}
    \label{sec:model}

    Our model achieved an AUC ROC score of:

    \[0.8752\]

    Taking in consideration that the new population profile does no match the
    one for which our model was trainned for it is still a high score.

    We can also have a look at the output of the model for the samples
    independently of our knowledge of the outcome.
    Ideally our models should give a loot more outcomes near 0 and 1 than
    in the middle.
    So lets look at the histogram of our \emph{proba} output in
    \vref{fig:proba}.


    While most of our outputs are near the edge, we see that this is only
    happening towards 1.
    This could mean that our new samples are a skewed view of the population
    or that the population itself is skewed toward 1 (becoming unemployed
    during the 12 moth period after the sample is collected).
    So lets compare with the histogram for the samples for which we do know
    the outcome (\vref{fig:proba-targets}) and the true outcomes
    (\vref{fig:proba-true}).

    \begin{figure}[!ht]
        \caption{Outcome and \emph{proba} histograms.}
        \begin{subfigure}[ht]{.5\linewidth}
            \subcaption{New samples \emph{proba}.}
            \label{fig:proba}
            \centering
            \includegraphics[width=\textwidth]{./img/proba.png}
        \end{subfigure}
        %
        \begin{subfigure}[ht]{.5\linewidth}
            \subcaption{New samples with outcomes \emph{proba}.}
            \label{fig:proba-targets}
            \centering
            \includegraphics[width=\textwidth]{./img/proba-targets.png}
        \end{subfigure}

        \vspace{5mm}

        \begin{subfigure}[!ht]{.5\linewidth}
            \subcaption{New outcomes.}
            \label{fig:proba-true}
            \centering
            \includegraphics[width=\textwidth]{./img/proba-true.png}
        \end{subfigure}
        %
        \begin{subfigure}[!ht]{.5\linewidth}
            \subcaption{Original outcomes.}
            \centering
            \includegraphics[width=\textwidth]{./img/orig-true.png}
        \end{subfigure}
    \end{figure}

    We can see that we are not in an ideal situation, our population outcomes
    are skewed, we have to be extra careful.
    But the AUC ROC score and the RUC curve (\vref{fig:roc}) tells us that we
    should have a good predictive capability by carefully selecting a
    threshold.

    \begin{figure}[!ht]
        \caption{ROC curve.}
        \label{fig:roc}
        \centering
        \includegraphics[width=\textwidth]{./img/roc.png}
    \end{figure}

    \section{Retraining}
    \label{sec:retrain}

    We joined the new data (with outcomes) with the old and split into
    training and test sets.
    Afterward we trained the same model (GradientBoostingClassifier) with the
    new data to compare the score with our deployed estimator, see
    \vref{tab:scores}.
    \begin{table}[!h]
        \caption{Country of origin top 10.}
        \label{tab:scores}

        \centering
        \subcaption{Original dataset.}
        \begin{tabular}{cc}
            Deployed estimator & 0.9182 \\
            New estimator & 0.9232 \\
        \end{tabular}
    \end{table}

    As we can see there is not a big difference, but there is so little new
    data that this just means that no new "insights" came from the new data.

    \appendix
    \section{Tables \& Figures}
\label{sec:tables}

\begin{table}[!h]
    \caption{Country of origin top 10.}
    \label{tab:country-of-origin}

    \begin{subtable}[ht]{.5\textwidth}
    \centering
    \subcaption{Original dataset.}
    \begin{tabular}{cc}
        u.s. & 7330 \\
        unknown & 126 \\
        mexico & 111 \\
        philippines & 60 \\
        de & 50 \\
        puerto rico & 39 \\
        jamaica & 34 \\
        cuba & 34 \\
        el-salvador & 30 \\
        canada & 28 \\
    \end{tabular}
    \end{subtable}
    %
    \begin{subtable}[ht]{.5\textwidth}
    \centering
    \subcaption{New dataset.}
    \begin{tabular}{cc}
        u.s. & 8935 \\
        mexico & 210 \\
        unknown & 185 \\
        philippines & 53 \\
        de & 39 \\
        india & 37 \\
        canada & 32 \\
        gb & 32 \\
        puerto rico & 30 \\
        el-salvador & 29 \\
    \end{tabular}
    \end{subtable}

    \vspace{5mm}

    \begin{subtable}[ht]{.5\textwidth}
    \centering
    \subcaption{New dataset samples with target.}
    \begin{tabular}{cc}
        u.s. & 452 \\
        mexico & 9 \\
        unknown & 8 \\
        de & 4 \\
        india & 3 \\
        hong & 2 \\
        puerto rico & 2 \\
        poland & 2 \\
        china & 2 \\
        ireland & 2 \\
    \end{tabular}
    \end{subtable}
\end{table}


\begin{figure}[!ht]
    \caption{Birth date histograms.}
    \begin{subfigure}[ht]{.5\linewidth}
        \subcaption{Original samples.}
        \label{fig:birth-date-hist-original}
        \centering
        \includegraphics[width=\textwidth]{./img/birth-date-freq-original.png}
    \end{subfigure}
    %
    \begin{subfigure}[ht]{.5\linewidth}
        \subcaption{New samples.}
        \label{fig:birth-date-hist-new}
        \centering
        \includegraphics[width=\textwidth]{./img/birth-date-freq-new.png}
    \end{subfigure}

    \vspace{5mm}

    \begin{subfigure}[!ht]{.5\linewidth}
        \subcaption{New samples with outcomes.}
        \label{fig:birth-date-hist-targets}
        \centering
        \includegraphics[width=\textwidth]{./img/birth-date-freq-targets.png}
    \end{subfigure}
\end{figure}


\begin{table}[!h]
    \caption{Domestic relationship type value counts.}
    \label{tab:domestic-relationship-type}

    \begin{subtable}[ht]{.5\textwidth}
        \centering
        \subcaption{Original dataset.}
        \begin{tabular}{cc}
            not living with family & 2919 \\
            never married & 2063 \\
            living with child & 1750 \\
            has husband & 1106 \\
            living with extende family & 325 \\
            has wife & 1 \\
        \end{tabular}
    \end{subtable}
    %
    \begin{subtable}[ht]{.5\textwidth}
        \centering
        \subcaption{New dataset.}
        \begin{tabular}{cc}
            has wife & 5398 \\
            not living with family & 2227 \\
            living with child & 1353 \\
            never married & 532 \\
            living with extende family & 245 \\
            has husband & 188 \\
        \end{tabular}
    \end{subtable}

    \vspace{5mm}

    \begin{subtable}[ht]{.5\textwidth}
        \centering
        \subcaption{New dataset samples with target.}
        \begin{tabular}{cc}
            has wife & 266 \\
            not living with family & 119 \\
            living with child & 66 \\
            never married & 23 \\
            living with extende family & 14 \\
            has husband & 10 \\
        \end{tabular}
    \end{subtable}
\end{table}


\begin{table}[!h]
    \caption{Domestic status value counts.}
    \label{tab:domestic-status}

    \begin{subtable}[ht]{.5\textwidth}
        \centering
        \subcaption{Original dataset.}
        \begin{tabular}{cc}
            single & 3662 \\
            d & 2073 \\
            married 2 & 1170 \\
            spouse passed & 599 \\
            divorce pending & 486 \\
            married not together & 163 \\
            married 1 & 11 \\
        \end{tabular}
    \end{subtable}
    %
    \begin{subtable}[ht]{.5\textwidth}
        \centering
        \subcaption{New dataset.}
        \begin{tabular}{cc}
            married 2 & 5645 \\
            single & 2853 \\
            d & 983 \\
            divorce pending & 211 \\
            spouse passed & 154 \\
            married not together & 93 \\
            married 1 & 4 \\
        \end{tabular}
    \end{subtable}

    \vspace{5mm}

    \begin{subtable}[ht]{.5\textwidth}
        \centering
        \subcaption{New dataset samples with target.}
        \begin{tabular}{cc}
            married 2 & 281 \\
            single & 145 \\
            d & 52 \\
            spouse passed & 8 \\
            married not together & 6 \\
            divorce pending & 6 \\
        \end{tabular}
    \end{subtable}
\end{table}


\begin{figure}[!ht]
    \caption{Earned dividends histograms.}
    \label{fig:earned-dividents-hist}

    \begin{subfigure}[ht]{.5\linewidth}
        \subcaption{Original samples.}
        \centering
        \includegraphics[width=\textwidth]{./img/earned-dividends-hist-original.png}
    \end{subfigure}
    %
    \begin{subfigure}[ht]{.5\linewidth}
        \subcaption{New samples.}
        \centering
        \includegraphics[width=\textwidth]{./img/earned-dividends-hist-new.png}
    \end{subfigure}

    \vspace{5mm}

    \begin{subfigure}[!ht]{.5\linewidth}
        \subcaption{New samples with outcomes.}
        \centering
        \includegraphics[width=\textwidth]{./img/earned-dividends-hist-targets.png}
    \end{subfigure}
\end{figure}


\begin{table}[!h]
    \caption{Ethnicity value counts.}
    \label{tab:ethnicity}

    \begin{subtable}[ht]{.5\textwidth}
        \centering
        \subcaption{Original dataset.}
        \begin{tabular}{cc}
            white and privileged & 6523 \\
            afro american & 1210 \\
            asian & 262 \\
            american indian & 88 \\
            other & 81 \\
        \end{tabular}
    \end{subtable}
    %
    \begin{subtable}[ht]{.5\textwidth}
        \centering
        \subcaption{New dataset.}
        \begin{tabular}{cc}
            white and privileged & 8679 \\
            afro american & 778 \\
            asian & 315 \\
            american indian & 92 \\
            other & 79 \\
        \end{tabular}
    \end{subtable}

    \vspace{5mm}

    \begin{subtable}[ht]{.5\textwidth}
        \centering
        \subcaption{New dataset samples with target.}
        \begin{tabular}{cc}
            white and privileged & 431 \\
            afro american & 46 \\
            asian & 17 \\
            american indian & 2 \\
            other & 2 \\
        \end{tabular}
    \end{subtable}
\end{table}


\begin{table}[!h]
    \caption{Gender value counts.}
    \label{tab:gender}

    \begin{subtable}[ht]{.5\textwidth}
        \centering
        \subcaption{Original dataset.}
        \begin{tabular}{cc}
            Female & 8164 \\
        \end{tabular}
    \end{subtable}
    %
    \begin{subtable}[ht]{.5\textwidth}
        \centering
        \subcaption{New dataset.}
        \begin{tabular}{cc}
            Male & 8904 \\
            Female & 1039 \\
        \end{tabular}
    \end{subtable}

    \vspace{5mm}

    \begin{subtable}[ht]{.5\textwidth}
        \centering
        \subcaption{New dataset samples with target.}
        \begin{tabular}{cc}
            Male & 453 \\
            Female & 45 \\
        \end{tabular}
    \end{subtable}
\end{table}


\begin{table}[!h]
    \caption{Job type value counts.}
    \label{tab:job-type}

    \begin{subtable}[ht]{.5\textwidth}
        \centering
        \subcaption{Original dataset.}
        \begin{tabular}{cc}
            private & 5919 \\
            unknown & 620 \\
            local-gov & 618 \\
            state-gov & 368 \\
            self-emp-not-inc & 303 \\
            federal-gov & 236 \\
            self-emp-inc & 94 \\
            without-pay & 4 \\
            never-worked & 2 \\
        \end{tabular}
    \end{subtable}
    %
    \begin{subtable}[ht]{.5\textwidth}
        \centering
        \subcaption{New dataset.}
        \begin{tabular}{cc}
            private & 6824 \\
            self-emp-not-inc & 905 \\
            local-gov & 573 \\
            unknown & 489 \\
            self-emp-inc & 426 \\
            state-gov & 415 \\
            federal-gov & 304 \\
            without-pay & 4 \\
            never-worked & 3 \\
        \end{tabular}
    \end{subtable}

    \vspace{5mm}

    \begin{subtable}[ht]{.5\textwidth}
        \centering
        \subcaption{New dataset samples with target.}
        \begin{tabular}{cc}
            private & 341 \\
            self-emp-not-inc & 44 \\
            unknown & 28 \\
            local-gov & 28 \\
            self-emp-inc & 24 \\
            state-gov & 20 \\
            federal-gov & 13 \\
        \end{tabular}
    \end{subtable}
\end{table}


\begin{figure}[!ht]
    \caption{Interest earned histograms.}
    \label{fig:interest-earned-hist}

    \begin{subfigure}[ht]{.5\linewidth}
        \subcaption{Original samples.}
        \centering
        \includegraphics[width=\textwidth]{./img/interest-earned-original.png}
    \end{subfigure}
    %
    \begin{subfigure}[ht]{.5\linewidth}
        \subcaption{New samples.}
        \centering
        \includegraphics[width=\textwidth]{./img/interest-earned-new.png}
    \end{subfigure}

    \vspace{5mm}

    \begin{subfigure}[!ht]{.5\linewidth}
        \subcaption{New samples with outcomes.}
        \centering
        \includegraphics[width=\textwidth]{./img/interest-earned-targets.png}
    \end{subfigure}
\end{figure}


\begin{figure}[!ht]
    \caption{Monthly work histograms.}
    \label{fig:monthly-work-hist}

    \begin{subfigure}[ht]{.5\linewidth}
        \subcaption{Original samples.}
        \centering
        \includegraphics[width=\textwidth]{./img/monthly-work-original.png}
    \end{subfigure}
    %
    \begin{subfigure}[ht]{.5\linewidth}
        \subcaption{New samples.}
        \centering
        \includegraphics[width=\textwidth]{./img/monthly-work-new.png}
    \end{subfigure}

    \vspace{5mm}

    \begin{subfigure}[!ht]{.5\linewidth}
        \subcaption{New samples with outcomes.}
        \centering
        \includegraphics[width=\textwidth]{./img/monthly-work-targets.png}
    \end{subfigure}
\end{figure}


\begin{table}[!h]
    \caption{Profession value counts.}
    \label{tab:profession}

    \begin{subtable}[ht]{.5\textwidth}
        \centering
        \subcaption{Original dataset.}
        \begin{tabular}{cc}
            secretarial & 1949 \\
            other & 1423 \\
            specialist technician & 1096 \\
            sales & 978 \\
            C-level & 842 \\
            unknown & 622 \\
            mechanic & 420 \\
            technology support & 247 \\
            vocational & 184 \\
            household labor & 131 \\
            estate employee & 108 \\
            defense contractor & 58 \\
            trucking & 58 \\
            agriculture & 48 \\
        \end{tabular}
    \end{subtable}
    %
    \begin{subtable}[ht]{.5\textwidth}
        \centering
        \subcaption{New dataset.}
        \begin{tabular}{cc}
            vocational & 1561 \\
            C-level & 1316 \\
            specialist technician & 1213 \\
            sales & 1112 \\
            other & 780 \\
            secretarial & 736 \\
            trucking & 666 \\
            mechanic & 637 \\
            household labor & 501 \\
            unknown & 492 \\
            agriculture & 388 \\
            technology support & 279 \\
            defense contractor & 244 \\
            estate employee & 16 \\
            army & 2 \\
        \end{tabular}
    \end{subtable}

    \vspace{5mm}

    \begin{subtable}[ht]{.5\textwidth}
        \centering
        \subcaption{New dataset samples with target.}
        \begin{tabular}{cc}
            C-level & 68 \\
            vocational & 68 \\
            sales & 64 \\
            specialist technician & 49 \\
            other & 42 \\
            trucking & 38 \\
            household labor & 33 \\
            mechanic & 30 \\
            secretarial & 29 \\
            unknown & 28 \\
            defense contractor & 18 \\
            agriculture & 17 \\
            technology support & 13 \\
            army & 1 \\
        \end{tabular}
    \end{subtable}
\end{table}


\begin{table}[!h]
    \caption{School level value counts.}
    \label{tab:}

    \begin{subtable}[ht]{.5\textwidth}
        \centering
        \subcaption{Original dataset.}
        \begin{tabular}{cc}
            secondary & 2594 \\
            entry level college & 2165 \\
            college graduate & 1188 \\
            basic vocational & 373 \\
            some post graduate & 355 \\
            secondary 11 & 341 \\
            advanced vocational & 326 \\
            10th & 248 \\
            secondary-7 through 8 & 123 \\
            secondary 12 & 106 \\
            secondary-9 & 104 \\
            secondary-5 through 6 & 72 \\
            advanced post graduate & 61 \\
            primary school & 58 \\
            primary 1 through 4 & 37 \\
            kindergarten & 13 \\
        \end{tabular}
    \end{subtable}
    %
    \begin{subtable}[ht]{.5\textwidth}
        \centering
        \subcaption{New dataset.}
        \begin{tabular}{cc}
            secondary & 3243 \\
            entry level college & 2087 \\
            college graduate & 1691 \\
            some post graduate & 577 \\
            basic vocational & 402 \\
            secondary 11 & 317 \\
            advanced vocational & 303 \\
            10th & 277 \\
            secondary-7 through 8 & 208 \\
            primary school & 202 \\
            secondary-9 & 172 \\
            advanced post graduate & 142 \\
            secondary 12 & 140 \\
            secondary-5 through 6 & 117 \\
            primary 1 through 4 & 49 \\
            kindergarten & 16 \\
        \end{tabular}
    \end{subtable}

    \vspace{5mm}

    \begin{subtable}[ht]{.5\textwidth}
        \centering
        \subcaption{New dataset samples with target.}
        \begin{tabular}{cc}
            secondary & 179 \\
            entry level college & 110 \\
            college graduate & 76 \\
            some post graduate & 26 \\
            secondary 11 & 20 \\
            primary school & 13 \\
            10th & 12 \\
            advanced vocational & 11 \\
            secondary 12 & 10 \\
            secondary-9 & 9 \\
            basic vocational & 9 \\
            advanced post graduate & 8 \\
            secondary-5 through 6 & 5 \\
            secondary-7 through 8 & 5 \\
            kindergarten & 3 \\
            primary 1 through 4 & 2 \\
        \end{tabular}
    \end{subtable}
\end{table}


%\bibliographystyle{IEEEtran}
%\bibliography{}
\end{document}