\section{Exploratory Data Analysis}
\label{sec:eda}

The dataset has 8164 samples with the following properties:
\begin{itemize}
\item Id
\item Birth date
\item Job type
\item School level
\item Domestic status
\item Profession
\item Domestic relationship type
\item Ethnicity
\item Gender
\item Earned dividends
\item Interest earned
\item Monthly work
\item Country of origin
\item Target
\end{itemize}
Each of the properties was analysed for insights and anomalies which are
described in the following sections.

\subsubsection*{Id}
Id for each subject, all values distinct not problems detected in the
dataset.

\subsubsection*{Birth date}
Date of birth of the subject, the youngest is from
\DTMdisplaydate{2016}{2}{10}{-1} and the oldest
\DTMdisplaydate{1928}{01}{09}{-1}.
Observing Figure \ref{fig:birth_date_freq} nothing surprinsing is detected.

\begin{figure}[ht]
    \caption{Date of birth frequency.}
    \label{fig:birth_date_freq}
    \centering
    %% Creator: Matplotlib, PGF backend
%%
%% To include the figure in your LaTeX document, write
%%   \input{<filename>.pgf}
%%
%% Make sure the required packages are loaded in your preamble
%%   \usepackage{pgf}
%%
%% Figures using additional raster images can only be included by \input if
%% they are in the same directory as the main LaTeX file. For loading figures
%% from other directories you can use the `import` package
%%   \usepackage{import}
%% and then include the figures with
%%   \import{<path to file>}{<filename>.pgf}
%%
%% Matplotlib used the following preamble
%%   \usepackage{fontspec}
%%   \setmainfont{DejaVu Serif}
%%   \setsansfont{DejaVu Sans}
%%   \setmonofont{DejaVu Sans Mono}
%%
\begingroup%
\makeatletter%
\begin{pgfpicture}%
\pgfpathrectangle{\pgfpointorigin}{\pgfqpoint{5.000000in}{3.900000in}}%
\pgfusepath{use as bounding box, clip}%
\begin{pgfscope}%
\pgfsetbuttcap%
\pgfsetmiterjoin%
\definecolor{currentfill}{rgb}{1.000000,1.000000,1.000000}%
\pgfsetfillcolor{currentfill}%
\pgfsetlinewidth{0.000000pt}%
\definecolor{currentstroke}{rgb}{1.000000,1.000000,1.000000}%
\pgfsetstrokecolor{currentstroke}%
\pgfsetdash{}{0pt}%
\pgfpathmoveto{\pgfqpoint{0.000000in}{0.000000in}}%
\pgfpathlineto{\pgfqpoint{5.000000in}{0.000000in}}%
\pgfpathlineto{\pgfqpoint{5.000000in}{3.900000in}}%
\pgfpathlineto{\pgfqpoint{0.000000in}{3.900000in}}%
\pgfpathclose%
\pgfusepath{fill}%
\end{pgfscope}%
\begin{pgfscope}%
\pgfsetbuttcap%
\pgfsetmiterjoin%
\definecolor{currentfill}{rgb}{1.000000,1.000000,1.000000}%
\pgfsetfillcolor{currentfill}%
\pgfsetlinewidth{0.000000pt}%
\definecolor{currentstroke}{rgb}{0.000000,0.000000,0.000000}%
\pgfsetstrokecolor{currentstroke}%
\pgfsetstrokeopacity{0.000000}%
\pgfsetdash{}{0pt}%
\pgfpathmoveto{\pgfqpoint{0.625000in}{0.429000in}}%
\pgfpathlineto{\pgfqpoint{4.500000in}{0.429000in}}%
\pgfpathlineto{\pgfqpoint{4.500000in}{3.432000in}}%
\pgfpathlineto{\pgfqpoint{0.625000in}{3.432000in}}%
\pgfpathclose%
\pgfusepath{fill}%
\end{pgfscope}%
\begin{pgfscope}%
\pgfsetbuttcap%
\pgfsetroundjoin%
\definecolor{currentfill}{rgb}{0.000000,0.000000,0.000000}%
\pgfsetfillcolor{currentfill}%
\pgfsetlinewidth{0.803000pt}%
\definecolor{currentstroke}{rgb}{0.000000,0.000000,0.000000}%
\pgfsetstrokecolor{currentstroke}%
\pgfsetdash{}{0pt}%
\pgfsys@defobject{currentmarker}{\pgfqpoint{0.000000in}{-0.048611in}}{\pgfqpoint{0.000000in}{0.000000in}}{%
\pgfpathmoveto{\pgfqpoint{0.000000in}{0.000000in}}%
\pgfpathlineto{\pgfqpoint{0.000000in}{-0.048611in}}%
\pgfusepath{stroke,fill}%
}%
\begin{pgfscope}%
\pgfsys@transformshift{0.898990in}{0.429000in}%
\pgfsys@useobject{currentmarker}{}%
\end{pgfscope}%
\end{pgfscope}%
\begin{pgfscope}%
\pgftext[x=0.898990in,y=0.331778in,,top]{\sffamily\fontsize{10.000000}{12.000000}\selectfont 1930}%
\end{pgfscope}%
\begin{pgfscope}%
\pgfsetbuttcap%
\pgfsetroundjoin%
\definecolor{currentfill}{rgb}{0.000000,0.000000,0.000000}%
\pgfsetfillcolor{currentfill}%
\pgfsetlinewidth{0.803000pt}%
\definecolor{currentstroke}{rgb}{0.000000,0.000000,0.000000}%
\pgfsetstrokecolor{currentstroke}%
\pgfsetdash{}{0pt}%
\pgfsys@defobject{currentmarker}{\pgfqpoint{0.000000in}{-0.048611in}}{\pgfqpoint{0.000000in}{0.000000in}}{%
\pgfpathmoveto{\pgfqpoint{0.000000in}{0.000000in}}%
\pgfpathlineto{\pgfqpoint{0.000000in}{-0.048611in}}%
\pgfusepath{stroke,fill}%
}%
\begin{pgfscope}%
\pgfsys@transformshift{1.388258in}{0.429000in}%
\pgfsys@useobject{currentmarker}{}%
\end{pgfscope}%
\end{pgfscope}%
\begin{pgfscope}%
\pgftext[x=1.388258in,y=0.331778in,,top]{\sffamily\fontsize{10.000000}{12.000000}\selectfont 1940}%
\end{pgfscope}%
\begin{pgfscope}%
\pgfsetbuttcap%
\pgfsetroundjoin%
\definecolor{currentfill}{rgb}{0.000000,0.000000,0.000000}%
\pgfsetfillcolor{currentfill}%
\pgfsetlinewidth{0.803000pt}%
\definecolor{currentstroke}{rgb}{0.000000,0.000000,0.000000}%
\pgfsetstrokecolor{currentstroke}%
\pgfsetdash{}{0pt}%
\pgfsys@defobject{currentmarker}{\pgfqpoint{0.000000in}{-0.048611in}}{\pgfqpoint{0.000000in}{0.000000in}}{%
\pgfpathmoveto{\pgfqpoint{0.000000in}{0.000000in}}%
\pgfpathlineto{\pgfqpoint{0.000000in}{-0.048611in}}%
\pgfusepath{stroke,fill}%
}%
\begin{pgfscope}%
\pgfsys@transformshift{1.877525in}{0.429000in}%
\pgfsys@useobject{currentmarker}{}%
\end{pgfscope}%
\end{pgfscope}%
\begin{pgfscope}%
\pgftext[x=1.877525in,y=0.331778in,,top]{\sffamily\fontsize{10.000000}{12.000000}\selectfont 1950}%
\end{pgfscope}%
\begin{pgfscope}%
\pgfsetbuttcap%
\pgfsetroundjoin%
\definecolor{currentfill}{rgb}{0.000000,0.000000,0.000000}%
\pgfsetfillcolor{currentfill}%
\pgfsetlinewidth{0.803000pt}%
\definecolor{currentstroke}{rgb}{0.000000,0.000000,0.000000}%
\pgfsetstrokecolor{currentstroke}%
\pgfsetdash{}{0pt}%
\pgfsys@defobject{currentmarker}{\pgfqpoint{0.000000in}{-0.048611in}}{\pgfqpoint{0.000000in}{0.000000in}}{%
\pgfpathmoveto{\pgfqpoint{0.000000in}{0.000000in}}%
\pgfpathlineto{\pgfqpoint{0.000000in}{-0.048611in}}%
\pgfusepath{stroke,fill}%
}%
\begin{pgfscope}%
\pgfsys@transformshift{2.366793in}{0.429000in}%
\pgfsys@useobject{currentmarker}{}%
\end{pgfscope}%
\end{pgfscope}%
\begin{pgfscope}%
\pgftext[x=2.366793in,y=0.331778in,,top]{\sffamily\fontsize{10.000000}{12.000000}\selectfont 1960}%
\end{pgfscope}%
\begin{pgfscope}%
\pgfsetbuttcap%
\pgfsetroundjoin%
\definecolor{currentfill}{rgb}{0.000000,0.000000,0.000000}%
\pgfsetfillcolor{currentfill}%
\pgfsetlinewidth{0.803000pt}%
\definecolor{currentstroke}{rgb}{0.000000,0.000000,0.000000}%
\pgfsetstrokecolor{currentstroke}%
\pgfsetdash{}{0pt}%
\pgfsys@defobject{currentmarker}{\pgfqpoint{0.000000in}{-0.048611in}}{\pgfqpoint{0.000000in}{0.000000in}}{%
\pgfpathmoveto{\pgfqpoint{0.000000in}{0.000000in}}%
\pgfpathlineto{\pgfqpoint{0.000000in}{-0.048611in}}%
\pgfusepath{stroke,fill}%
}%
\begin{pgfscope}%
\pgfsys@transformshift{2.856061in}{0.429000in}%
\pgfsys@useobject{currentmarker}{}%
\end{pgfscope}%
\end{pgfscope}%
\begin{pgfscope}%
\pgftext[x=2.856061in,y=0.331778in,,top]{\sffamily\fontsize{10.000000}{12.000000}\selectfont 1970}%
\end{pgfscope}%
\begin{pgfscope}%
\pgfsetbuttcap%
\pgfsetroundjoin%
\definecolor{currentfill}{rgb}{0.000000,0.000000,0.000000}%
\pgfsetfillcolor{currentfill}%
\pgfsetlinewidth{0.803000pt}%
\definecolor{currentstroke}{rgb}{0.000000,0.000000,0.000000}%
\pgfsetstrokecolor{currentstroke}%
\pgfsetdash{}{0pt}%
\pgfsys@defobject{currentmarker}{\pgfqpoint{0.000000in}{-0.048611in}}{\pgfqpoint{0.000000in}{0.000000in}}{%
\pgfpathmoveto{\pgfqpoint{0.000000in}{0.000000in}}%
\pgfpathlineto{\pgfqpoint{0.000000in}{-0.048611in}}%
\pgfusepath{stroke,fill}%
}%
\begin{pgfscope}%
\pgfsys@transformshift{3.345328in}{0.429000in}%
\pgfsys@useobject{currentmarker}{}%
\end{pgfscope}%
\end{pgfscope}%
\begin{pgfscope}%
\pgftext[x=3.345328in,y=0.331778in,,top]{\sffamily\fontsize{10.000000}{12.000000}\selectfont 1980}%
\end{pgfscope}%
\begin{pgfscope}%
\pgfsetbuttcap%
\pgfsetroundjoin%
\definecolor{currentfill}{rgb}{0.000000,0.000000,0.000000}%
\pgfsetfillcolor{currentfill}%
\pgfsetlinewidth{0.803000pt}%
\definecolor{currentstroke}{rgb}{0.000000,0.000000,0.000000}%
\pgfsetstrokecolor{currentstroke}%
\pgfsetdash{}{0pt}%
\pgfsys@defobject{currentmarker}{\pgfqpoint{0.000000in}{-0.048611in}}{\pgfqpoint{0.000000in}{0.000000in}}{%
\pgfpathmoveto{\pgfqpoint{0.000000in}{0.000000in}}%
\pgfpathlineto{\pgfqpoint{0.000000in}{-0.048611in}}%
\pgfusepath{stroke,fill}%
}%
\begin{pgfscope}%
\pgfsys@transformshift{3.834596in}{0.429000in}%
\pgfsys@useobject{currentmarker}{}%
\end{pgfscope}%
\end{pgfscope}%
\begin{pgfscope}%
\pgftext[x=3.834596in,y=0.331778in,,top]{\sffamily\fontsize{10.000000}{12.000000}\selectfont 1990}%
\end{pgfscope}%
\begin{pgfscope}%
\pgfsetbuttcap%
\pgfsetroundjoin%
\definecolor{currentfill}{rgb}{0.000000,0.000000,0.000000}%
\pgfsetfillcolor{currentfill}%
\pgfsetlinewidth{0.803000pt}%
\definecolor{currentstroke}{rgb}{0.000000,0.000000,0.000000}%
\pgfsetstrokecolor{currentstroke}%
\pgfsetdash{}{0pt}%
\pgfsys@defobject{currentmarker}{\pgfqpoint{0.000000in}{-0.048611in}}{\pgfqpoint{0.000000in}{0.000000in}}{%
\pgfpathmoveto{\pgfqpoint{0.000000in}{0.000000in}}%
\pgfpathlineto{\pgfqpoint{0.000000in}{-0.048611in}}%
\pgfusepath{stroke,fill}%
}%
\begin{pgfscope}%
\pgfsys@transformshift{4.323864in}{0.429000in}%
\pgfsys@useobject{currentmarker}{}%
\end{pgfscope}%
\end{pgfscope}%
\begin{pgfscope}%
\pgftext[x=4.323864in,y=0.331778in,,top]{\sffamily\fontsize{10.000000}{12.000000}\selectfont 2000}%
\end{pgfscope}%
\begin{pgfscope}%
\pgftext[x=2.562500in,y=0.141809in,,top]{\sffamily\fontsize{10.000000}{12.000000}\selectfont birth date}%
\end{pgfscope}%
\begin{pgfscope}%
\pgfsetbuttcap%
\pgfsetroundjoin%
\definecolor{currentfill}{rgb}{0.000000,0.000000,0.000000}%
\pgfsetfillcolor{currentfill}%
\pgfsetlinewidth{0.803000pt}%
\definecolor{currentstroke}{rgb}{0.000000,0.000000,0.000000}%
\pgfsetstrokecolor{currentstroke}%
\pgfsetdash{}{0pt}%
\pgfsys@defobject{currentmarker}{\pgfqpoint{-0.048611in}{0.000000in}}{\pgfqpoint{0.000000in}{0.000000in}}{%
\pgfpathmoveto{\pgfqpoint{0.000000in}{0.000000in}}%
\pgfpathlineto{\pgfqpoint{-0.048611in}{0.000000in}}%
\pgfusepath{stroke,fill}%
}%
\begin{pgfscope}%
\pgfsys@transformshift{0.625000in}{0.555887in}%
\pgfsys@useobject{currentmarker}{}%
\end{pgfscope}%
\end{pgfscope}%
\begin{pgfscope}%
\pgftext[x=0.439412in,y=0.503126in,left,base]{\sffamily\fontsize{10.000000}{12.000000}\selectfont 0}%
\end{pgfscope}%
\begin{pgfscope}%
\pgfsetbuttcap%
\pgfsetroundjoin%
\definecolor{currentfill}{rgb}{0.000000,0.000000,0.000000}%
\pgfsetfillcolor{currentfill}%
\pgfsetlinewidth{0.803000pt}%
\definecolor{currentstroke}{rgb}{0.000000,0.000000,0.000000}%
\pgfsetstrokecolor{currentstroke}%
\pgfsetdash{}{0pt}%
\pgfsys@defobject{currentmarker}{\pgfqpoint{-0.048611in}{0.000000in}}{\pgfqpoint{0.000000in}{0.000000in}}{%
\pgfpathmoveto{\pgfqpoint{0.000000in}{0.000000in}}%
\pgfpathlineto{\pgfqpoint{-0.048611in}{0.000000in}}%
\pgfusepath{stroke,fill}%
}%
\begin{pgfscope}%
\pgfsys@transformshift{0.625000in}{1.036521in}%
\pgfsys@useobject{currentmarker}{}%
\end{pgfscope}%
\end{pgfscope}%
\begin{pgfscope}%
\pgftext[x=0.351047in,y=0.983760in,left,base]{\sffamily\fontsize{10.000000}{12.000000}\selectfont 50}%
\end{pgfscope}%
\begin{pgfscope}%
\pgfsetbuttcap%
\pgfsetroundjoin%
\definecolor{currentfill}{rgb}{0.000000,0.000000,0.000000}%
\pgfsetfillcolor{currentfill}%
\pgfsetlinewidth{0.803000pt}%
\definecolor{currentstroke}{rgb}{0.000000,0.000000,0.000000}%
\pgfsetstrokecolor{currentstroke}%
\pgfsetdash{}{0pt}%
\pgfsys@defobject{currentmarker}{\pgfqpoint{-0.048611in}{0.000000in}}{\pgfqpoint{0.000000in}{0.000000in}}{%
\pgfpathmoveto{\pgfqpoint{0.000000in}{0.000000in}}%
\pgfpathlineto{\pgfqpoint{-0.048611in}{0.000000in}}%
\pgfusepath{stroke,fill}%
}%
\begin{pgfscope}%
\pgfsys@transformshift{0.625000in}{1.517155in}%
\pgfsys@useobject{currentmarker}{}%
\end{pgfscope}%
\end{pgfscope}%
\begin{pgfscope}%
\pgftext[x=0.262682in,y=1.464393in,left,base]{\sffamily\fontsize{10.000000}{12.000000}\selectfont 100}%
\end{pgfscope}%
\begin{pgfscope}%
\pgfsetbuttcap%
\pgfsetroundjoin%
\definecolor{currentfill}{rgb}{0.000000,0.000000,0.000000}%
\pgfsetfillcolor{currentfill}%
\pgfsetlinewidth{0.803000pt}%
\definecolor{currentstroke}{rgb}{0.000000,0.000000,0.000000}%
\pgfsetstrokecolor{currentstroke}%
\pgfsetdash{}{0pt}%
\pgfsys@defobject{currentmarker}{\pgfqpoint{-0.048611in}{0.000000in}}{\pgfqpoint{0.000000in}{0.000000in}}{%
\pgfpathmoveto{\pgfqpoint{0.000000in}{0.000000in}}%
\pgfpathlineto{\pgfqpoint{-0.048611in}{0.000000in}}%
\pgfusepath{stroke,fill}%
}%
\begin{pgfscope}%
\pgfsys@transformshift{0.625000in}{1.997789in}%
\pgfsys@useobject{currentmarker}{}%
\end{pgfscope}%
\end{pgfscope}%
\begin{pgfscope}%
\pgftext[x=0.262682in,y=1.945027in,left,base]{\sffamily\fontsize{10.000000}{12.000000}\selectfont 150}%
\end{pgfscope}%
\begin{pgfscope}%
\pgfsetbuttcap%
\pgfsetroundjoin%
\definecolor{currentfill}{rgb}{0.000000,0.000000,0.000000}%
\pgfsetfillcolor{currentfill}%
\pgfsetlinewidth{0.803000pt}%
\definecolor{currentstroke}{rgb}{0.000000,0.000000,0.000000}%
\pgfsetstrokecolor{currentstroke}%
\pgfsetdash{}{0pt}%
\pgfsys@defobject{currentmarker}{\pgfqpoint{-0.048611in}{0.000000in}}{\pgfqpoint{0.000000in}{0.000000in}}{%
\pgfpathmoveto{\pgfqpoint{0.000000in}{0.000000in}}%
\pgfpathlineto{\pgfqpoint{-0.048611in}{0.000000in}}%
\pgfusepath{stroke,fill}%
}%
\begin{pgfscope}%
\pgfsys@transformshift{0.625000in}{2.478423in}%
\pgfsys@useobject{currentmarker}{}%
\end{pgfscope}%
\end{pgfscope}%
\begin{pgfscope}%
\pgftext[x=0.262682in,y=2.425661in,left,base]{\sffamily\fontsize{10.000000}{12.000000}\selectfont 200}%
\end{pgfscope}%
\begin{pgfscope}%
\pgfsetbuttcap%
\pgfsetroundjoin%
\definecolor{currentfill}{rgb}{0.000000,0.000000,0.000000}%
\pgfsetfillcolor{currentfill}%
\pgfsetlinewidth{0.803000pt}%
\definecolor{currentstroke}{rgb}{0.000000,0.000000,0.000000}%
\pgfsetstrokecolor{currentstroke}%
\pgfsetdash{}{0pt}%
\pgfsys@defobject{currentmarker}{\pgfqpoint{-0.048611in}{0.000000in}}{\pgfqpoint{0.000000in}{0.000000in}}{%
\pgfpathmoveto{\pgfqpoint{0.000000in}{0.000000in}}%
\pgfpathlineto{\pgfqpoint{-0.048611in}{0.000000in}}%
\pgfusepath{stroke,fill}%
}%
\begin{pgfscope}%
\pgfsys@transformshift{0.625000in}{2.959056in}%
\pgfsys@useobject{currentmarker}{}%
\end{pgfscope}%
\end{pgfscope}%
\begin{pgfscope}%
\pgftext[x=0.262682in,y=2.906295in,left,base]{\sffamily\fontsize{10.000000}{12.000000}\selectfont 250}%
\end{pgfscope}%
\begin{pgfscope}%
\pgfpathrectangle{\pgfqpoint{0.625000in}{0.429000in}}{\pgfqpoint{3.875000in}{3.003000in}} %
\pgfusepath{clip}%
\pgfsetrectcap%
\pgfsetroundjoin%
\pgfsetlinewidth{1.505625pt}%
\definecolor{currentstroke}{rgb}{0.121569,0.466667,0.705882}%
\pgfsetstrokecolor{currentstroke}%
\pgfsetdash{}{0pt}%
\pgfpathmoveto{\pgfqpoint{0.801136in}{0.661627in}}%
\pgfpathlineto{\pgfqpoint{0.898990in}{0.565500in}}%
\pgfpathlineto{\pgfqpoint{0.996843in}{0.565500in}}%
\pgfpathlineto{\pgfqpoint{1.094697in}{0.584725in}}%
\pgfpathlineto{\pgfqpoint{1.143624in}{0.565500in}}%
\pgfpathlineto{\pgfqpoint{1.192551in}{0.575113in}}%
\pgfpathlineto{\pgfqpoint{1.241477in}{0.603951in}}%
\pgfpathlineto{\pgfqpoint{1.290404in}{0.584725in}}%
\pgfpathlineto{\pgfqpoint{1.339331in}{0.584725in}}%
\pgfpathlineto{\pgfqpoint{1.388258in}{0.584725in}}%
\pgfpathlineto{\pgfqpoint{1.437184in}{0.632789in}}%
\pgfpathlineto{\pgfqpoint{1.486111in}{0.652014in}}%
\pgfpathlineto{\pgfqpoint{1.535038in}{0.652014in}}%
\pgfpathlineto{\pgfqpoint{1.583965in}{0.690465in}}%
\pgfpathlineto{\pgfqpoint{1.632891in}{0.728915in}}%
\pgfpathlineto{\pgfqpoint{1.681818in}{0.709690in}}%
\pgfpathlineto{\pgfqpoint{1.730745in}{0.738528in}}%
\pgfpathlineto{\pgfqpoint{1.779672in}{0.690465in}}%
\pgfpathlineto{\pgfqpoint{1.828598in}{0.719303in}}%
\pgfpathlineto{\pgfqpoint{1.877525in}{0.815430in}}%
\pgfpathlineto{\pgfqpoint{1.926452in}{0.805817in}}%
\pgfpathlineto{\pgfqpoint{1.975379in}{0.882718in}}%
\pgfpathlineto{\pgfqpoint{2.024306in}{0.950007in}}%
\pgfpathlineto{\pgfqpoint{2.073232in}{1.074972in}}%
\pgfpathlineto{\pgfqpoint{2.122159in}{0.959620in}}%
\pgfpathlineto{\pgfqpoint{2.171086in}{1.209549in}}%
\pgfpathlineto{\pgfqpoint{2.220013in}{1.171099in}}%
\pgfpathlineto{\pgfqpoint{2.268939in}{1.132648in}}%
\pgfpathlineto{\pgfqpoint{2.317866in}{1.296063in}}%
\pgfpathlineto{\pgfqpoint{2.366793in}{1.978563in}}%
\pgfpathlineto{\pgfqpoint{2.415720in}{1.305676in}}%
\pgfpathlineto{\pgfqpoint{2.464646in}{1.526768in}}%
\pgfpathlineto{\pgfqpoint{2.513573in}{1.296063in}}%
\pgfpathlineto{\pgfqpoint{2.562500in}{1.517155in}}%
\pgfpathlineto{\pgfqpoint{2.611427in}{1.497930in}}%
\pgfpathlineto{\pgfqpoint{2.660354in}{1.757472in}}%
\pgfpathlineto{\pgfqpoint{2.709280in}{1.776697in}}%
\pgfpathlineto{\pgfqpoint{2.758207in}{1.834373in}}%
\pgfpathlineto{\pgfqpoint{2.807134in}{1.824761in}}%
\pgfpathlineto{\pgfqpoint{2.856061in}{2.199655in}}%
\pgfpathlineto{\pgfqpoint{2.904987in}{2.257331in}}%
\pgfpathlineto{\pgfqpoint{2.953914in}{2.180430in}}%
\pgfpathlineto{\pgfqpoint{3.002841in}{2.113141in}}%
\pgfpathlineto{\pgfqpoint{3.051768in}{2.266944in}}%
\pgfpathlineto{\pgfqpoint{3.100694in}{2.132366in}}%
\pgfpathlineto{\pgfqpoint{3.149621in}{2.305394in}}%
\pgfpathlineto{\pgfqpoint{3.198548in}{2.266944in}}%
\pgfpathlineto{\pgfqpoint{3.247475in}{2.315007in}}%
\pgfpathlineto{\pgfqpoint{3.296402in}{2.170817in}}%
\pgfpathlineto{\pgfqpoint{3.345328in}{2.180430in}}%
\pgfpathlineto{\pgfqpoint{3.394255in}{2.728352in}}%
\pgfpathlineto{\pgfqpoint{3.443182in}{2.420746in}}%
\pgfpathlineto{\pgfqpoint{3.492109in}{2.497648in}}%
\pgfpathlineto{\pgfqpoint{3.541035in}{2.372683in}}%
\pgfpathlineto{\pgfqpoint{3.589962in}{2.391908in}}%
\pgfpathlineto{\pgfqpoint{3.638889in}{2.536099in}}%
\pgfpathlineto{\pgfqpoint{3.687816in}{2.574549in}}%
\pgfpathlineto{\pgfqpoint{3.736742in}{2.382296in}}%
\pgfpathlineto{\pgfqpoint{3.785669in}{2.766803in}}%
\pgfpathlineto{\pgfqpoint{3.834596in}{2.814866in}}%
\pgfpathlineto{\pgfqpoint{3.883523in}{2.747577in}}%
\pgfpathlineto{\pgfqpoint{3.932449in}{2.843704in}}%
\pgfpathlineto{\pgfqpoint{3.981376in}{2.795641in}}%
\pgfpathlineto{\pgfqpoint{4.030303in}{3.189761in}}%
\pgfpathlineto{\pgfqpoint{4.079230in}{2.997507in}}%
\pgfpathlineto{\pgfqpoint{4.128157in}{3.007120in}}%
\pgfpathlineto{\pgfqpoint{4.177083in}{3.247437in}}%
\pgfpathlineto{\pgfqpoint{4.226010in}{3.295500in}}%
\pgfpathlineto{\pgfqpoint{4.274937in}{2.747577in}}%
\pgfpathlineto{\pgfqpoint{4.323864in}{1.940113in}}%
\pgfusepath{stroke}%
\end{pgfscope}%
\begin{pgfscope}%
\pgfsetrectcap%
\pgfsetmiterjoin%
\pgfsetlinewidth{0.803000pt}%
\definecolor{currentstroke}{rgb}{0.000000,0.000000,0.000000}%
\pgfsetstrokecolor{currentstroke}%
\pgfsetdash{}{0pt}%
\pgfpathmoveto{\pgfqpoint{0.625000in}{0.429000in}}%
\pgfpathlineto{\pgfqpoint{0.625000in}{3.432000in}}%
\pgfusepath{stroke}%
\end{pgfscope}%
\begin{pgfscope}%
\pgfsetrectcap%
\pgfsetmiterjoin%
\pgfsetlinewidth{0.803000pt}%
\definecolor{currentstroke}{rgb}{0.000000,0.000000,0.000000}%
\pgfsetstrokecolor{currentstroke}%
\pgfsetdash{}{0pt}%
\pgfpathmoveto{\pgfqpoint{4.500000in}{0.429000in}}%
\pgfpathlineto{\pgfqpoint{4.500000in}{3.432000in}}%
\pgfusepath{stroke}%
\end{pgfscope}%
\begin{pgfscope}%
\pgfsetrectcap%
\pgfsetmiterjoin%
\pgfsetlinewidth{0.803000pt}%
\definecolor{currentstroke}{rgb}{0.000000,0.000000,0.000000}%
\pgfsetstrokecolor{currentstroke}%
\pgfsetdash{}{0pt}%
\pgfpathmoveto{\pgfqpoint{0.625000in}{0.429000in}}%
\pgfpathlineto{\pgfqpoint{4.500000in}{0.429000in}}%
\pgfusepath{stroke}%
\end{pgfscope}%
\begin{pgfscope}%
\pgfsetrectcap%
\pgfsetmiterjoin%
\pgfsetlinewidth{0.803000pt}%
\definecolor{currentstroke}{rgb}{0.000000,0.000000,0.000000}%
\pgfsetstrokecolor{currentstroke}%
\pgfsetdash{}{0pt}%
\pgfpathmoveto{\pgfqpoint{0.625000in}{3.432000in}}%
\pgfpathlineto{\pgfqpoint{4.500000in}{3.432000in}}%
\pgfusepath{stroke}%
\end{pgfscope}%
\end{pgfpicture}%
\makeatother%
\endgroup%

\end{figure}

\subsubsection*{Job type}
The current job type of the subject, government, self employed, etc\ldots
Value counts are available in Table \ref{tab:job_type_value_count}.

\begin{table}
    \caption{Job type value counts.}
    \label{tab:job_type_value_count}
    \centering
    \begin{tabular}{cc}
        private & 5919 \\
        unknown & 620 \\
        local-gov & 618 \\
        state-gov & 368 \\
        self-emp-not-inc & 303 \\
        federal-gov & 236 \\
        self-emp-inc & 94 \\
        without-pay & 4 \\
        never-worked & 2 \\
    \end{tabular}
\end{table}

