\section{Modelling}
\label{sec:modelling}

\section{Initial tests}
\label{sec:initialTests}
To get a feelling of the baseline performance of the models available in
the \emph{scikit} package severall were tried with the default parameters
(except were not possible), see \vref{tab:initialModelScores}.

\begin{table}[!h]
    \caption{}
    \label{tab:initialModelScores}
    \centering
    \begin{tabular}{cc}
        Model & AUC ROC Score \\
        \hline
        GradientBoostingClassifier & 0.9212 \\
        AdaBoostClassifier & 0.9008 \\
        BaggingClassifier & 0.8428 \\
        VotingClassifier & 0.8233 \\
        RandomForestClassifier & 0.8160 \\
        ExtraTreesClassifier & 0.8154 \\
        QuadraticDiscriminantAnalysis & 0.8053 \\
        KNeighborsClassifier & 0.7267 \\
        GaussianProcessClassifier & 0.7022 \\
        DecisionTreeClassifier & 0.6729 \\
        SVC & 0.6698 \\
        SGDClassifier & 0.6377 \\
        GaussianNB & 0.6718 \\
    \end{tabular}
\end{table}


With the default parameters there is, as expected, a clear dominance of
ensemble models.
The top 5 were selected for further parameter tuning.

\section{Model tuning \& selection}
\label{sec:modelTuning}

Each of the models was optimized by randomly searching a small part of the
parameter space.
The portion to explore was determined empirically by careful study of
each of the parameter.
The results are detailed in \vref{tab:hyperModelScores}.

\begin{table}[!h]
    \caption{}
    \label{tab:hyperModelScores}
    \centering
    \begin{tabular}{cc}
        Model & AUC ROC Score \\
        \hline
        GradientBoostingClassifier & 0.9369 \\
        AdaBoostClassifier & 0.9329 \\
        RandomForestClassifier & 0.9308 \\
        VotingClassifier & 0.9299 \\
        BaggingClassifier & 0.9143 \\
    \end{tabular}
\end{table}

After tuning all models were able to achieve AUR ROC scores in the .9 range,
but \emph{GradientBoostingClassifier} outperformed the others.
As there are no other constraints model selection is based solely on the score.


\section{Feature Elimination}
\label{sec:featureElimination}

Running recursive feature elimination on the model identified in the previous
section the optimal number of features was determined to be 31 and are the
following:
\begin{itemize}
    \item birth date
    \item interest earned
    \item monthly work
    \item job type--federal-gov
    \item job type--self-emp-not-inc
    \item school level--10th
    \item school level--advanced post graduate
    \item school level--college graduate
    \item school level--primary school
    \item school level--secondary
    \item school level--some post graduate
    \item domestic status--married 1
    \item domestic status--married 2
    \item domestic status--spouse passed
    \item profession--C-level
    \item profession--defense contractor
    \item profession--mechanic
    \item profession--other
    \item profession--secretarial
    \item profession--specialist technician
    \item profession--trucking
    \item profession--vocational
    \item domestic relationship type--has husband
    \item domestic relationship type--not living with family
    \item ethnicity--afro american
    \item country of origin--GR
    \item country of origin--HU
    \item country of origin--IE
    \item country of origin--JP
    \item country of origin--PH
    \item country of origin--US
\end{itemize}

\begin{figure}[!h]
    \caption{Score change.}
    \label{fig:score-features}
    \centering
    %% Creator: Matplotlib, PGF backend
%%
%% To include the figure in your LaTeX document, write
%%   \input{<filename>.pgf}
%%
%% Make sure the required packages are loaded in your preamble
%%   \usepackage{pgf}
%%
%% Figures using additional raster images can only be included by \input if
%% they are in the same directory as the main LaTeX file. For loading figures
%% from other directories you can use the `import` package
%%   \usepackage{import}
%% and then include the figures with
%%   \import{<path to file>}{<filename>.pgf}
%%
%% Matplotlib used the following preamble
%%   \usepackage{fontspec}
%%   \setmainfont{DejaVu Serif}
%%   \setsansfont{DejaVu Sans}
%%   \setmonofont{DejaVu Sans Mono}
%%
\begingroup%
\makeatletter%
\begin{pgfpicture}%
\pgfpathrectangle{\pgfpointorigin}{\pgfqpoint{6.400000in}{4.800000in}}%
\pgfusepath{use as bounding box, clip}%
\begin{pgfscope}%
\pgfsetbuttcap%
\pgfsetmiterjoin%
\definecolor{currentfill}{rgb}{1.000000,1.000000,1.000000}%
\pgfsetfillcolor{currentfill}%
\pgfsetlinewidth{0.000000pt}%
\definecolor{currentstroke}{rgb}{1.000000,1.000000,1.000000}%
\pgfsetstrokecolor{currentstroke}%
\pgfsetdash{}{0pt}%
\pgfpathmoveto{\pgfqpoint{0.000000in}{0.000000in}}%
\pgfpathlineto{\pgfqpoint{6.400000in}{0.000000in}}%
\pgfpathlineto{\pgfqpoint{6.400000in}{4.800000in}}%
\pgfpathlineto{\pgfqpoint{0.000000in}{4.800000in}}%
\pgfpathclose%
\pgfusepath{fill}%
\end{pgfscope}%
\begin{pgfscope}%
\pgfsetbuttcap%
\pgfsetmiterjoin%
\definecolor{currentfill}{rgb}{1.000000,1.000000,1.000000}%
\pgfsetfillcolor{currentfill}%
\pgfsetlinewidth{0.000000pt}%
\definecolor{currentstroke}{rgb}{0.000000,0.000000,0.000000}%
\pgfsetstrokecolor{currentstroke}%
\pgfsetstrokeopacity{0.000000}%
\pgfsetdash{}{0pt}%
\pgfpathmoveto{\pgfqpoint{0.800000in}{0.528000in}}%
\pgfpathlineto{\pgfqpoint{5.760000in}{0.528000in}}%
\pgfpathlineto{\pgfqpoint{5.760000in}{4.224000in}}%
\pgfpathlineto{\pgfqpoint{0.800000in}{4.224000in}}%
\pgfpathclose%
\pgfusepath{fill}%
\end{pgfscope}%
\begin{pgfscope}%
\pgfsetbuttcap%
\pgfsetroundjoin%
\definecolor{currentfill}{rgb}{0.000000,0.000000,0.000000}%
\pgfsetfillcolor{currentfill}%
\pgfsetlinewidth{0.803000pt}%
\definecolor{currentstroke}{rgb}{0.000000,0.000000,0.000000}%
\pgfsetstrokecolor{currentstroke}%
\pgfsetdash{}{0pt}%
\pgfsys@defobject{currentmarker}{\pgfqpoint{0.000000in}{-0.048611in}}{\pgfqpoint{0.000000in}{0.000000in}}{%
\pgfpathmoveto{\pgfqpoint{0.000000in}{0.000000in}}%
\pgfpathlineto{\pgfqpoint{0.000000in}{-0.048611in}}%
\pgfusepath{stroke,fill}%
}%
\begin{pgfscope}%
\pgfsys@transformshift{0.980364in}{0.528000in}%
\pgfsys@useobject{currentmarker}{}%
\end{pgfscope}%
\end{pgfscope}%
\begin{pgfscope}%
\pgftext[x=0.980364in,y=0.430778in,,top]{\sffamily\fontsize{10.000000}{12.000000}\selectfont 0}%
\end{pgfscope}%
\begin{pgfscope}%
\pgfsetbuttcap%
\pgfsetroundjoin%
\definecolor{currentfill}{rgb}{0.000000,0.000000,0.000000}%
\pgfsetfillcolor{currentfill}%
\pgfsetlinewidth{0.803000pt}%
\definecolor{currentstroke}{rgb}{0.000000,0.000000,0.000000}%
\pgfsetstrokecolor{currentstroke}%
\pgfsetdash{}{0pt}%
\pgfsys@defobject{currentmarker}{\pgfqpoint{0.000000in}{-0.048611in}}{\pgfqpoint{0.000000in}{0.000000in}}{%
\pgfpathmoveto{\pgfqpoint{0.000000in}{0.000000in}}%
\pgfpathlineto{\pgfqpoint{0.000000in}{-0.048611in}}%
\pgfusepath{stroke,fill}%
}%
\begin{pgfscope}%
\pgfsys@transformshift{1.882182in}{0.528000in}%
\pgfsys@useobject{currentmarker}{}%
\end{pgfscope}%
\end{pgfscope}%
\begin{pgfscope}%
\pgftext[x=1.882182in,y=0.430778in,,top]{\sffamily\fontsize{10.000000}{12.000000}\selectfont 20}%
\end{pgfscope}%
\begin{pgfscope}%
\pgfsetbuttcap%
\pgfsetroundjoin%
\definecolor{currentfill}{rgb}{0.000000,0.000000,0.000000}%
\pgfsetfillcolor{currentfill}%
\pgfsetlinewidth{0.803000pt}%
\definecolor{currentstroke}{rgb}{0.000000,0.000000,0.000000}%
\pgfsetstrokecolor{currentstroke}%
\pgfsetdash{}{0pt}%
\pgfsys@defobject{currentmarker}{\pgfqpoint{0.000000in}{-0.048611in}}{\pgfqpoint{0.000000in}{0.000000in}}{%
\pgfpathmoveto{\pgfqpoint{0.000000in}{0.000000in}}%
\pgfpathlineto{\pgfqpoint{0.000000in}{-0.048611in}}%
\pgfusepath{stroke,fill}%
}%
\begin{pgfscope}%
\pgfsys@transformshift{2.784000in}{0.528000in}%
\pgfsys@useobject{currentmarker}{}%
\end{pgfscope}%
\end{pgfscope}%
\begin{pgfscope}%
\pgftext[x=2.784000in,y=0.430778in,,top]{\sffamily\fontsize{10.000000}{12.000000}\selectfont 40}%
\end{pgfscope}%
\begin{pgfscope}%
\pgfsetbuttcap%
\pgfsetroundjoin%
\definecolor{currentfill}{rgb}{0.000000,0.000000,0.000000}%
\pgfsetfillcolor{currentfill}%
\pgfsetlinewidth{0.803000pt}%
\definecolor{currentstroke}{rgb}{0.000000,0.000000,0.000000}%
\pgfsetstrokecolor{currentstroke}%
\pgfsetdash{}{0pt}%
\pgfsys@defobject{currentmarker}{\pgfqpoint{0.000000in}{-0.048611in}}{\pgfqpoint{0.000000in}{0.000000in}}{%
\pgfpathmoveto{\pgfqpoint{0.000000in}{0.000000in}}%
\pgfpathlineto{\pgfqpoint{0.000000in}{-0.048611in}}%
\pgfusepath{stroke,fill}%
}%
\begin{pgfscope}%
\pgfsys@transformshift{3.685818in}{0.528000in}%
\pgfsys@useobject{currentmarker}{}%
\end{pgfscope}%
\end{pgfscope}%
\begin{pgfscope}%
\pgftext[x=3.685818in,y=0.430778in,,top]{\sffamily\fontsize{10.000000}{12.000000}\selectfont 60}%
\end{pgfscope}%
\begin{pgfscope}%
\pgfsetbuttcap%
\pgfsetroundjoin%
\definecolor{currentfill}{rgb}{0.000000,0.000000,0.000000}%
\pgfsetfillcolor{currentfill}%
\pgfsetlinewidth{0.803000pt}%
\definecolor{currentstroke}{rgb}{0.000000,0.000000,0.000000}%
\pgfsetstrokecolor{currentstroke}%
\pgfsetdash{}{0pt}%
\pgfsys@defobject{currentmarker}{\pgfqpoint{0.000000in}{-0.048611in}}{\pgfqpoint{0.000000in}{0.000000in}}{%
\pgfpathmoveto{\pgfqpoint{0.000000in}{0.000000in}}%
\pgfpathlineto{\pgfqpoint{0.000000in}{-0.048611in}}%
\pgfusepath{stroke,fill}%
}%
\begin{pgfscope}%
\pgfsys@transformshift{4.587636in}{0.528000in}%
\pgfsys@useobject{currentmarker}{}%
\end{pgfscope}%
\end{pgfscope}%
\begin{pgfscope}%
\pgftext[x=4.587636in,y=0.430778in,,top]{\sffamily\fontsize{10.000000}{12.000000}\selectfont 80}%
\end{pgfscope}%
\begin{pgfscope}%
\pgfsetbuttcap%
\pgfsetroundjoin%
\definecolor{currentfill}{rgb}{0.000000,0.000000,0.000000}%
\pgfsetfillcolor{currentfill}%
\pgfsetlinewidth{0.803000pt}%
\definecolor{currentstroke}{rgb}{0.000000,0.000000,0.000000}%
\pgfsetstrokecolor{currentstroke}%
\pgfsetdash{}{0pt}%
\pgfsys@defobject{currentmarker}{\pgfqpoint{0.000000in}{-0.048611in}}{\pgfqpoint{0.000000in}{0.000000in}}{%
\pgfpathmoveto{\pgfqpoint{0.000000in}{0.000000in}}%
\pgfpathlineto{\pgfqpoint{0.000000in}{-0.048611in}}%
\pgfusepath{stroke,fill}%
}%
\begin{pgfscope}%
\pgfsys@transformshift{5.489455in}{0.528000in}%
\pgfsys@useobject{currentmarker}{}%
\end{pgfscope}%
\end{pgfscope}%
\begin{pgfscope}%
\pgftext[x=5.489455in,y=0.430778in,,top]{\sffamily\fontsize{10.000000}{12.000000}\selectfont 100}%
\end{pgfscope}%
\begin{pgfscope}%
\pgftext[x=3.280000in,y=0.240809in,,top]{\sffamily\fontsize{10.000000}{12.000000}\selectfont Number of features selected}%
\end{pgfscope}%
\begin{pgfscope}%
\pgfsetbuttcap%
\pgfsetroundjoin%
\definecolor{currentfill}{rgb}{0.000000,0.000000,0.000000}%
\pgfsetfillcolor{currentfill}%
\pgfsetlinewidth{0.803000pt}%
\definecolor{currentstroke}{rgb}{0.000000,0.000000,0.000000}%
\pgfsetstrokecolor{currentstroke}%
\pgfsetdash{}{0pt}%
\pgfsys@defobject{currentmarker}{\pgfqpoint{-0.048611in}{0.000000in}}{\pgfqpoint{0.000000in}{0.000000in}}{%
\pgfpathmoveto{\pgfqpoint{0.000000in}{0.000000in}}%
\pgfpathlineto{\pgfqpoint{-0.048611in}{0.000000in}}%
\pgfusepath{stroke,fill}%
}%
\begin{pgfscope}%
\pgfsys@transformshift{0.800000in}{0.779616in}%
\pgfsys@useobject{currentmarker}{}%
\end{pgfscope}%
\end{pgfscope}%
\begin{pgfscope}%
\pgftext[x=0.393533in,y=0.726855in,left,base]{\sffamily\fontsize{10.000000}{12.000000}\selectfont 0.60}%
\end{pgfscope}%
\begin{pgfscope}%
\pgfsetbuttcap%
\pgfsetroundjoin%
\definecolor{currentfill}{rgb}{0.000000,0.000000,0.000000}%
\pgfsetfillcolor{currentfill}%
\pgfsetlinewidth{0.803000pt}%
\definecolor{currentstroke}{rgb}{0.000000,0.000000,0.000000}%
\pgfsetstrokecolor{currentstroke}%
\pgfsetdash{}{0pt}%
\pgfsys@defobject{currentmarker}{\pgfqpoint{-0.048611in}{0.000000in}}{\pgfqpoint{0.000000in}{0.000000in}}{%
\pgfpathmoveto{\pgfqpoint{0.000000in}{0.000000in}}%
\pgfpathlineto{\pgfqpoint{-0.048611in}{0.000000in}}%
\pgfusepath{stroke,fill}%
}%
\begin{pgfscope}%
\pgfsys@transformshift{0.800000in}{1.294728in}%
\pgfsys@useobject{currentmarker}{}%
\end{pgfscope}%
\end{pgfscope}%
\begin{pgfscope}%
\pgftext[x=0.393533in,y=1.241967in,left,base]{\sffamily\fontsize{10.000000}{12.000000}\selectfont 0.65}%
\end{pgfscope}%
\begin{pgfscope}%
\pgfsetbuttcap%
\pgfsetroundjoin%
\definecolor{currentfill}{rgb}{0.000000,0.000000,0.000000}%
\pgfsetfillcolor{currentfill}%
\pgfsetlinewidth{0.803000pt}%
\definecolor{currentstroke}{rgb}{0.000000,0.000000,0.000000}%
\pgfsetstrokecolor{currentstroke}%
\pgfsetdash{}{0pt}%
\pgfsys@defobject{currentmarker}{\pgfqpoint{-0.048611in}{0.000000in}}{\pgfqpoint{0.000000in}{0.000000in}}{%
\pgfpathmoveto{\pgfqpoint{0.000000in}{0.000000in}}%
\pgfpathlineto{\pgfqpoint{-0.048611in}{0.000000in}}%
\pgfusepath{stroke,fill}%
}%
\begin{pgfscope}%
\pgfsys@transformshift{0.800000in}{1.809840in}%
\pgfsys@useobject{currentmarker}{}%
\end{pgfscope}%
\end{pgfscope}%
\begin{pgfscope}%
\pgftext[x=0.393533in,y=1.757078in,left,base]{\sffamily\fontsize{10.000000}{12.000000}\selectfont 0.70}%
\end{pgfscope}%
\begin{pgfscope}%
\pgfsetbuttcap%
\pgfsetroundjoin%
\definecolor{currentfill}{rgb}{0.000000,0.000000,0.000000}%
\pgfsetfillcolor{currentfill}%
\pgfsetlinewidth{0.803000pt}%
\definecolor{currentstroke}{rgb}{0.000000,0.000000,0.000000}%
\pgfsetstrokecolor{currentstroke}%
\pgfsetdash{}{0pt}%
\pgfsys@defobject{currentmarker}{\pgfqpoint{-0.048611in}{0.000000in}}{\pgfqpoint{0.000000in}{0.000000in}}{%
\pgfpathmoveto{\pgfqpoint{0.000000in}{0.000000in}}%
\pgfpathlineto{\pgfqpoint{-0.048611in}{0.000000in}}%
\pgfusepath{stroke,fill}%
}%
\begin{pgfscope}%
\pgfsys@transformshift{0.800000in}{2.324951in}%
\pgfsys@useobject{currentmarker}{}%
\end{pgfscope}%
\end{pgfscope}%
\begin{pgfscope}%
\pgftext[x=0.393533in,y=2.272190in,left,base]{\sffamily\fontsize{10.000000}{12.000000}\selectfont 0.75}%
\end{pgfscope}%
\begin{pgfscope}%
\pgfsetbuttcap%
\pgfsetroundjoin%
\definecolor{currentfill}{rgb}{0.000000,0.000000,0.000000}%
\pgfsetfillcolor{currentfill}%
\pgfsetlinewidth{0.803000pt}%
\definecolor{currentstroke}{rgb}{0.000000,0.000000,0.000000}%
\pgfsetstrokecolor{currentstroke}%
\pgfsetdash{}{0pt}%
\pgfsys@defobject{currentmarker}{\pgfqpoint{-0.048611in}{0.000000in}}{\pgfqpoint{0.000000in}{0.000000in}}{%
\pgfpathmoveto{\pgfqpoint{0.000000in}{0.000000in}}%
\pgfpathlineto{\pgfqpoint{-0.048611in}{0.000000in}}%
\pgfusepath{stroke,fill}%
}%
\begin{pgfscope}%
\pgfsys@transformshift{0.800000in}{2.840063in}%
\pgfsys@useobject{currentmarker}{}%
\end{pgfscope}%
\end{pgfscope}%
\begin{pgfscope}%
\pgftext[x=0.393533in,y=2.787302in,left,base]{\sffamily\fontsize{10.000000}{12.000000}\selectfont 0.80}%
\end{pgfscope}%
\begin{pgfscope}%
\pgfsetbuttcap%
\pgfsetroundjoin%
\definecolor{currentfill}{rgb}{0.000000,0.000000,0.000000}%
\pgfsetfillcolor{currentfill}%
\pgfsetlinewidth{0.803000pt}%
\definecolor{currentstroke}{rgb}{0.000000,0.000000,0.000000}%
\pgfsetstrokecolor{currentstroke}%
\pgfsetdash{}{0pt}%
\pgfsys@defobject{currentmarker}{\pgfqpoint{-0.048611in}{0.000000in}}{\pgfqpoint{0.000000in}{0.000000in}}{%
\pgfpathmoveto{\pgfqpoint{0.000000in}{0.000000in}}%
\pgfpathlineto{\pgfqpoint{-0.048611in}{0.000000in}}%
\pgfusepath{stroke,fill}%
}%
\begin{pgfscope}%
\pgfsys@transformshift{0.800000in}{3.355175in}%
\pgfsys@useobject{currentmarker}{}%
\end{pgfscope}%
\end{pgfscope}%
\begin{pgfscope}%
\pgftext[x=0.393533in,y=3.302413in,left,base]{\sffamily\fontsize{10.000000}{12.000000}\selectfont 0.85}%
\end{pgfscope}%
\begin{pgfscope}%
\pgfsetbuttcap%
\pgfsetroundjoin%
\definecolor{currentfill}{rgb}{0.000000,0.000000,0.000000}%
\pgfsetfillcolor{currentfill}%
\pgfsetlinewidth{0.803000pt}%
\definecolor{currentstroke}{rgb}{0.000000,0.000000,0.000000}%
\pgfsetstrokecolor{currentstroke}%
\pgfsetdash{}{0pt}%
\pgfsys@defobject{currentmarker}{\pgfqpoint{-0.048611in}{0.000000in}}{\pgfqpoint{0.000000in}{0.000000in}}{%
\pgfpathmoveto{\pgfqpoint{0.000000in}{0.000000in}}%
\pgfpathlineto{\pgfqpoint{-0.048611in}{0.000000in}}%
\pgfusepath{stroke,fill}%
}%
\begin{pgfscope}%
\pgfsys@transformshift{0.800000in}{3.870287in}%
\pgfsys@useobject{currentmarker}{}%
\end{pgfscope}%
\end{pgfscope}%
\begin{pgfscope}%
\pgftext[x=0.393533in,y=3.817525in,left,base]{\sffamily\fontsize{10.000000}{12.000000}\selectfont 0.90}%
\end{pgfscope}%
\begin{pgfscope}%
\pgftext[x=0.337977in,y=2.376000in,,bottom,rotate=90.000000]{\sffamily\fontsize{10.000000}{12.000000}\selectfont ROC AUC score}%
\end{pgfscope}%
\begin{pgfscope}%
\pgfpathrectangle{\pgfqpoint{0.800000in}{0.528000in}}{\pgfqpoint{4.960000in}{3.696000in}} %
\pgfusepath{clip}%
\pgfsetrectcap%
\pgfsetroundjoin%
\pgfsetlinewidth{1.505625pt}%
\definecolor{currentstroke}{rgb}{0.121569,0.466667,0.705882}%
\pgfsetstrokecolor{currentstroke}%
\pgfsetdash{}{0pt}%
\pgfpathmoveto{\pgfqpoint{1.025455in}{0.696000in}}%
\pgfpathlineto{\pgfqpoint{1.070545in}{1.525726in}}%
\pgfpathlineto{\pgfqpoint{1.115636in}{2.189283in}}%
\pgfpathlineto{\pgfqpoint{1.160727in}{3.774503in}}%
\pgfpathlineto{\pgfqpoint{1.205818in}{3.821087in}}%
\pgfpathlineto{\pgfqpoint{1.250909in}{3.866341in}}%
\pgfpathlineto{\pgfqpoint{1.296000in}{3.908342in}}%
\pgfpathlineto{\pgfqpoint{1.341091in}{3.934008in}}%
\pgfpathlineto{\pgfqpoint{1.386182in}{3.942832in}}%
\pgfpathlineto{\pgfqpoint{1.431273in}{3.947114in}}%
\pgfpathlineto{\pgfqpoint{1.476364in}{3.950300in}}%
\pgfpathlineto{\pgfqpoint{1.521455in}{3.984909in}}%
\pgfpathlineto{\pgfqpoint{1.566545in}{3.991830in}}%
\pgfpathlineto{\pgfqpoint{1.611636in}{3.977279in}}%
\pgfpathlineto{\pgfqpoint{1.656727in}{3.981097in}}%
\pgfpathlineto{\pgfqpoint{1.701818in}{3.992697in}}%
\pgfpathlineto{\pgfqpoint{1.746909in}{4.006061in}}%
\pgfpathlineto{\pgfqpoint{1.792000in}{4.024519in}}%
\pgfpathlineto{\pgfqpoint{1.837091in}{4.037851in}}%
\pgfpathlineto{\pgfqpoint{1.882182in}{4.047877in}}%
\pgfpathlineto{\pgfqpoint{1.927273in}{4.052190in}}%
\pgfpathlineto{\pgfqpoint{1.972364in}{4.045809in}}%
\pgfpathlineto{\pgfqpoint{2.017455in}{4.050931in}}%
\pgfpathlineto{\pgfqpoint{2.062545in}{4.049691in}}%
\pgfpathlineto{\pgfqpoint{2.107636in}{4.051929in}}%
\pgfpathlineto{\pgfqpoint{2.152727in}{4.054541in}}%
\pgfpathlineto{\pgfqpoint{2.197818in}{4.054274in}}%
\pgfpathlineto{\pgfqpoint{2.242909in}{4.044811in}}%
\pgfpathlineto{\pgfqpoint{2.288000in}{4.047536in}}%
\pgfpathlineto{\pgfqpoint{2.333091in}{4.045444in}}%
\pgfpathlineto{\pgfqpoint{2.378182in}{4.056000in}}%
\pgfpathlineto{\pgfqpoint{2.423273in}{4.045190in}}%
\pgfpathlineto{\pgfqpoint{2.468364in}{4.049923in}}%
\pgfpathlineto{\pgfqpoint{2.513455in}{4.045471in}}%
\pgfpathlineto{\pgfqpoint{2.558545in}{4.047203in}}%
\pgfpathlineto{\pgfqpoint{2.603636in}{4.046792in}}%
\pgfpathlineto{\pgfqpoint{2.648727in}{4.046896in}}%
\pgfpathlineto{\pgfqpoint{2.693818in}{4.045958in}}%
\pgfpathlineto{\pgfqpoint{2.738909in}{4.044714in}}%
\pgfpathlineto{\pgfqpoint{2.784000in}{4.045369in}}%
\pgfpathlineto{\pgfqpoint{2.829091in}{4.045369in}}%
\pgfpathlineto{\pgfqpoint{2.874182in}{4.045369in}}%
\pgfpathlineto{\pgfqpoint{2.919273in}{4.051476in}}%
\pgfpathlineto{\pgfqpoint{2.964364in}{4.051476in}}%
\pgfpathlineto{\pgfqpoint{3.009455in}{4.051476in}}%
\pgfpathlineto{\pgfqpoint{3.054545in}{4.051476in}}%
\pgfpathlineto{\pgfqpoint{3.099636in}{4.051476in}}%
\pgfpathlineto{\pgfqpoint{3.144727in}{4.051476in}}%
\pgfpathlineto{\pgfqpoint{3.189818in}{4.051476in}}%
\pgfpathlineto{\pgfqpoint{3.234909in}{4.051476in}}%
\pgfpathlineto{\pgfqpoint{3.280000in}{4.051476in}}%
\pgfpathlineto{\pgfqpoint{3.325091in}{4.051476in}}%
\pgfpathlineto{\pgfqpoint{3.370182in}{4.051476in}}%
\pgfpathlineto{\pgfqpoint{3.415273in}{4.051476in}}%
\pgfpathlineto{\pgfqpoint{3.460364in}{4.051476in}}%
\pgfpathlineto{\pgfqpoint{3.505455in}{4.051476in}}%
\pgfpathlineto{\pgfqpoint{3.550545in}{4.051476in}}%
\pgfpathlineto{\pgfqpoint{3.595636in}{4.051476in}}%
\pgfpathlineto{\pgfqpoint{3.640727in}{4.051476in}}%
\pgfpathlineto{\pgfqpoint{3.685818in}{4.051476in}}%
\pgfpathlineto{\pgfqpoint{3.730909in}{4.051476in}}%
\pgfpathlineto{\pgfqpoint{3.776000in}{4.051476in}}%
\pgfpathlineto{\pgfqpoint{3.821091in}{4.051476in}}%
\pgfpathlineto{\pgfqpoint{3.866182in}{4.051476in}}%
\pgfpathlineto{\pgfqpoint{3.911273in}{4.051476in}}%
\pgfpathlineto{\pgfqpoint{3.956364in}{4.051476in}}%
\pgfpathlineto{\pgfqpoint{4.001455in}{4.051476in}}%
\pgfpathlineto{\pgfqpoint{4.046545in}{4.051476in}}%
\pgfpathlineto{\pgfqpoint{4.091636in}{4.051476in}}%
\pgfpathlineto{\pgfqpoint{4.136727in}{4.051476in}}%
\pgfpathlineto{\pgfqpoint{4.181818in}{4.051476in}}%
\pgfpathlineto{\pgfqpoint{4.226909in}{4.051476in}}%
\pgfpathlineto{\pgfqpoint{4.272000in}{4.051476in}}%
\pgfpathlineto{\pgfqpoint{4.317091in}{4.051476in}}%
\pgfpathlineto{\pgfqpoint{4.362182in}{4.051476in}}%
\pgfpathlineto{\pgfqpoint{4.407273in}{4.051476in}}%
\pgfpathlineto{\pgfqpoint{4.452364in}{4.051476in}}%
\pgfpathlineto{\pgfqpoint{4.497455in}{4.051476in}}%
\pgfpathlineto{\pgfqpoint{4.542545in}{4.051476in}}%
\pgfpathlineto{\pgfqpoint{4.587636in}{4.051476in}}%
\pgfpathlineto{\pgfqpoint{4.632727in}{4.051476in}}%
\pgfpathlineto{\pgfqpoint{4.677818in}{4.051476in}}%
\pgfpathlineto{\pgfqpoint{4.722909in}{4.051476in}}%
\pgfpathlineto{\pgfqpoint{4.768000in}{4.051476in}}%
\pgfpathlineto{\pgfqpoint{4.813091in}{4.051476in}}%
\pgfpathlineto{\pgfqpoint{4.858182in}{4.051476in}}%
\pgfpathlineto{\pgfqpoint{4.903273in}{4.051476in}}%
\pgfpathlineto{\pgfqpoint{4.948364in}{4.051476in}}%
\pgfpathlineto{\pgfqpoint{4.993455in}{4.051476in}}%
\pgfpathlineto{\pgfqpoint{5.038545in}{4.051476in}}%
\pgfpathlineto{\pgfqpoint{5.083636in}{4.051476in}}%
\pgfpathlineto{\pgfqpoint{5.128727in}{4.051476in}}%
\pgfpathlineto{\pgfqpoint{5.173818in}{4.051476in}}%
\pgfpathlineto{\pgfqpoint{5.218909in}{4.051476in}}%
\pgfpathlineto{\pgfqpoint{5.264000in}{4.051476in}}%
\pgfpathlineto{\pgfqpoint{5.309091in}{4.051476in}}%
\pgfpathlineto{\pgfqpoint{5.354182in}{4.051476in}}%
\pgfpathlineto{\pgfqpoint{5.399273in}{4.051476in}}%
\pgfpathlineto{\pgfqpoint{5.444364in}{4.051476in}}%
\pgfpathlineto{\pgfqpoint{5.489455in}{4.051476in}}%
\pgfpathlineto{\pgfqpoint{5.534545in}{4.051476in}}%
\pgfusepath{stroke}%
\end{pgfscope}%
\begin{pgfscope}%
\pgfsetrectcap%
\pgfsetmiterjoin%
\pgfsetlinewidth{0.803000pt}%
\definecolor{currentstroke}{rgb}{0.000000,0.000000,0.000000}%
\pgfsetstrokecolor{currentstroke}%
\pgfsetdash{}{0pt}%
\pgfpathmoveto{\pgfqpoint{0.800000in}{0.528000in}}%
\pgfpathlineto{\pgfqpoint{0.800000in}{4.224000in}}%
\pgfusepath{stroke}%
\end{pgfscope}%
\begin{pgfscope}%
\pgfsetrectcap%
\pgfsetmiterjoin%
\pgfsetlinewidth{0.803000pt}%
\definecolor{currentstroke}{rgb}{0.000000,0.000000,0.000000}%
\pgfsetstrokecolor{currentstroke}%
\pgfsetdash{}{0pt}%
\pgfpathmoveto{\pgfqpoint{5.760000in}{0.528000in}}%
\pgfpathlineto{\pgfqpoint{5.760000in}{4.224000in}}%
\pgfusepath{stroke}%
\end{pgfscope}%
\begin{pgfscope}%
\pgfsetrectcap%
\pgfsetmiterjoin%
\pgfsetlinewidth{0.803000pt}%
\definecolor{currentstroke}{rgb}{0.000000,0.000000,0.000000}%
\pgfsetstrokecolor{currentstroke}%
\pgfsetdash{}{0pt}%
\pgfpathmoveto{\pgfqpoint{0.800000in}{0.528000in}}%
\pgfpathlineto{\pgfqpoint{5.760000in}{0.528000in}}%
\pgfusepath{stroke}%
\end{pgfscope}%
\begin{pgfscope}%
\pgfsetrectcap%
\pgfsetmiterjoin%
\pgfsetlinewidth{0.803000pt}%
\definecolor{currentstroke}{rgb}{0.000000,0.000000,0.000000}%
\pgfsetstrokecolor{currentstroke}%
\pgfsetdash{}{0pt}%
\pgfpathmoveto{\pgfqpoint{0.800000in}{4.224000in}}%
\pgfpathlineto{\pgfqpoint{5.760000in}{4.224000in}}%
\pgfusepath{stroke}%
\end{pgfscope}%
\end{pgfpicture}%
\makeatother%
\endgroup%

\end{figure}


\Vref{fig:score-features} displays how the AUC ROC score changes with  the number of features.
Since there are no computational limitations and the prediction does not
deteriorate with the number of features this exercise is a mere curiosity.

Is is interesting however to compare the selected features with their
correlation (see \vref{tab:feature-correlation}) with the target,
remember that a positive correlation means it more likely to end up
unemployed in 12 months.
The selected features that correlate positively with being unemployed
soon are presented in the following list.
Note that this list says nothing of the decision surface of our model,
having any combination of these features does not mean that you are
more likely to become unemployed.
\begin{itemize}
    \item birth date
    \item school level--10th
    \item school level--secondary
    \item domestic status--spouse passed
    \item profession--mechanic
    \item profession--other
    \item profession--secretarial
    \item profession--trucking
    \item profession--vocational
    \item domestic relationship type--not living with family
    \item ethnicity--afro american
    \item country of origin--US
\end{itemize}


\begin{table}[!h]
\caption{Selected features and respective correlation.}
\label{tab:feature-correlation}
\centering
\begin{tabular}{cc}
    birth date & 0.114102 \\
    interest earned & -0.148656 \\
    monthly work & -0.110180 \\
    job type--federal-gov & -0.042745 \\
    job type--self-emp-not-inc & -0.065876 \\
    school level--10th & 0.047029 \\
    school level--advanced post graduate & -0.139795 \\
    school level--college graduate & -0.129046 \\
    school level--primary school & -0.077406 \\
    school level--secondary & 0.083645 \\
    school level--some post graduate & -0.145206 \\
    domestic status--married 1 & -0.047914 \\
    domestic status--married 2 & -0.469246 \\
    domestic status--spouse passed & 0.050169 \\
    profession--C-level & -0.131167 \\
    profession--defense contractor & -0.010123 \\
    profession--mechanic & 0.046294 \\
    profession--other & 0.105133 \\
    profession--secretarial & 0.036197 \\
    profession--specialist technician & -0.171728 \\
    profession--trucking & 0.005404 \\
    profession--vocational & 0.011720 \\
    domestic relationship type--has husband & -0.481157 \\
    domestic relationship type--not living with family & 0.094195 \\
    ethnicity--afro american & 0.068792 \\
    country of origin--GR & -0.016763 \\
    country of origin--HU & -0.016763 \\
    country of origin--IE & -0.009930 \\
    country of origin--JP & -0.027360 \\
    country of origin--PH & -0.024337 \\
    country of origin--US & 0.004980 \\
\end{tabular}
\end{table}
