\section{Data pre-processing}
\label{sec:data}

This section details any and all data pre-processing before modelling.
The first section \vref{subsec:cleaning} explains what was done to
convert the dataset to usable, unambiguous types.
Afterwards section \vref{subsec:feature} details the what was
done to select and improve the features feed to the model.
Finally section \vref{subsec:splitting} breaks down the train/test split.


\subsection{Data cleaning}
\label{subsec:cleaning}

On data import the \emph{birth dates} were converted to naive dates as no
timezone information was provided.
It's doubtful the timezone would provide any useful information.

The \emph{countries of origin} were converted
to the corresponding \emph{ISO 3166--1 alpha-2} representation.
Some inputs required special rules.
Especially ambiguous was \emph{dr} which
represents no country code this was converted to Dominican Republic (DO)
even if the race for these inputs suggests it's not (mostly white).

All categorical data columns was kept as is, there was not enough information
to reach any conclusion.


\subsection{Feature engineering}
\label{subsec:feature}

\emph{birth dates} were converted to timestamps.
In the dataset earned dividends and gender do not change,
these properties were dropped since they convey no useful information.
If new samples include this value this decision will be reconsidered.
All categorical data was turned into dummy class variables.


\subsection{Train/test splitting}
\label{subsec:splitting}

The train/test split was done holding out .4 of the data for final
validation.
Furthermore training was done using a shuffle split with .3 for testing.

