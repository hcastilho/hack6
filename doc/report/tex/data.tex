\section{Data pre-processing}
\label{sec:data}

This section details any and all data pre-processing before modelling.
The first section \vref{subsec:cleaning} explains what was done to
convert the dataset to usable, unambiguous types.
Afterwards section \vref{subsec:feature} details the what was
done to select and improve the features feed to the model.
Finally section \vref{subsec:splitting} breaks down the train/test split.


\subsection{Data cleaning}
\label{subsec:cleaning}

% birth date
To input in the models birth dates where converted to timestamps.


% country of origin
On importing the dataset all countries where converted
to the corresponding \emph{ISO 3166--1 alpha-2} representation.
Some inputs required special rules.
Especially ambiguous was \emph{dr} which
represents no country code this was converted to Dominican Republic (DO)
even if the race for these inputs suggests it's not (mostly white).

% categorical data
For categorical data, the classes where kept as is and turned into dummy class
variables.



\subsection{Feature engineering}
\label{subsec:feature}

% Earned dividends
In the dataset earned dividends and gender do not change,
these properties where dropped since they convey no useful information.
If new samples include this value this decision will be reconsidered.


\subsection{Train/test splitting}
\label{subsec:splitting}

The train/test split was done holding out .4 of the data for final
validation (train{\_}test{\_}split).
Furthermore training was done using a shuffle split (\emph{ShuffleSplitCV})
with .3 for testing.

